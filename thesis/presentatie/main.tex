%-------------------------------------------------------------------------------
% LATEX TEMPLATE PRESENTATIE
%-------------------------------------------------------------------------------
% Dit template is voor gebruik door studenten van de de bacheloropleiding 
% Informatica van de Universiteit van Amsterdam.
% Voor informatie over presenteren, zie 
%                               https://practicumav.nl/presenteren/index.html
%
%-------------------------------------------------------------------------------
%	PACKAGES EN CONFIGURATIE
%-------------------------------------------------------------------------------
% Gebruik de optie "sidebar" voor toevoeging van een sidebar met inhoudsopgave
% en de optie "dyslexic" voor gebruik van het OpenDyslexic-lettertype
\documentclass[aspectratio=169,sidebar]{uva-inf-presentation}
\usepackage[english]{babel}
\usepackage{listings}

% Vul hieronder de gevraagde gegevens in, 
% meerdere auteurs en UvAnetID's gescheiden door een puntkomma 
\title{AI-Driven Retrieval and Analysis of Videotaped Council Meeting Archives using Automatic Speech Recognition, Speaker Identification, and Retrieval Augmented Generation} 
\authors{Pepijn van WIjk}
\uvanetids{13952072}
\course{Bachelor thesis}
\tutor{}
\docent{Dr. Maarten Marx} 
\group{}
\programme{Informatica}

\begin{document}
%-------------------------------------------------------------------------------
%	AUTOMATISCHE SLIDES
%-------------------------------------------------------------------------------
\begin{titelframe}
\titlepage
\end{titelframe}

\begin{frame}{Introduction}
    \begin{itemize}
        \item Wet Open Overheid
        \begin{itemize}
            \item \textbf{Active Publication}: Government bodies should actively publish certain types of information.
            \item \textbf{Publication on Request}: Information should be made public upon request.
            \item \textbf{Information Management Obligation}: All information should be easily accessible.
        \end{itemize}
        \item WooGle provides document search
        \item Example: journalist writing about a local meeting decision
    \end{itemize}
\end{frame}

\begin{titelframe}\frametitle{Contents}
\tableofcontents
\end{titelframe}

%-------------------------------------------------------------------------------
%	PRESENTATIE SLIDES
%-------------------------------------------------------------------------------


\section{Research questions}
% \subsection{Eerste subsectie}


\begin{frame}\frametitle{Main research question}\footnotesize
\begin{itemize}
    \item How can AI be utilized in order to increase information retrievability of large video archives, in particular of democratically elected councils?
\end{itemize}
\end{frame}

\begin{frame}\frametitle{Research questions}\footnotesize
\begin{enumerate}
    \item What are the different archive formats local authorities host in order to comply with the Woo and how are these formats exploited to retrieve as much information as possible?
    \item What state-of-the-art video- and audio analysis tools can be used to improve searchability of the meetings, and how accurate are these in the given circumstances?
    \item How accurately can such a video be segmented into the meaningfully different parts which are discussed?
    \item What state-of-the-art information retrieval techniques can be leveraged to develop an efficient video information retrieval system?
    \item How well can the developed search system be integrated with a large language model, creating a helpful chat bot capable of answering questions and providing complementary information when needed?
\end{enumerate}
\end{frame}


\section{Background information}
\begin{frame}\frametitle{Word embeddings}
\begin{itemize}
    \item Word feature vectors
    \begin{itemize}
        \item Encapsulate semantic meaning
    \end{itemize}
    \item Vector search
    \item Vector database    
    \item BM25 keyword search
\end{itemize}
\end{frame}

\begin{frame}{Word feature vector example}
    \begin{figure}
    \centering
    \includegraphics[width=0.85\textwidth]{images/embeddings.png}
    \caption{A two-dimensional (t-SNE) projection of word embeddings.}
    \label{fig:embeddings}
\end{figure}
\end{frame}


% \subsection{Retrieval Augmented Generation}
\begin{frame}\frametitle{Retrieval Augmented Generation}
\begin{itemize}
    \item Chat bot
    \item Large Language Models
    \begin{itemize}
        \item Vast factual information
        \item Limited by training data
    \end{itemize}
    \item \Knowledge Knowledge specific queries
    \begin{itemize}
        \item Require external information
        \item Vector database
    \end{itemize}
    \item Benefits
    \begin{itemize}
        \item More accurate answers
        \item Reduced hallucinations
    \end{itemize}
\end{itemize}
\end{frame}

\section{Methodology}
\begin{frame}{Archival Data Extraction}
    \begin{itemize}
        \item Research question 1
        \begin{itemize}
            \item Archive formats and information retrieval
        \end{itemize}
        \item Service Providers
        \begin{itemize}
            \item \href{https://www.ibabs.com}{iBabs}
            \item \href{https://www.notubiz.nl/}{NotuBiz}
        \end{itemize}
        \item Video \& agenda scrape tool
    \end{itemize}
\end{frame}

\begin{frame}{Video- and Audio Analysis}
    \begin{itemize}
        \item Research question 2
        \begin{itemize}
            \item State-of-the-art analysis tools
        \end{itemize}
        \item Automatic Speech Recognition (ASR)
        \begin{itemize}
            \item "What has been spoken?"
            \item Whisper
        \end{itemize}
        \item Speaker diarisation
        \begin{itemize}
            \item "Who spoke when?"
            \item pyannote.audio
        \end{itemize}
        \item Modular design
        \begin{itemize}
            \item Python class inheritance
            \item Modular and easily upgradable
        \end{itemize}
    \end{itemize}
\end{frame}

\begin{frame}[fragile]{Code Example: Diarisor Class}
\begin{lstlisting}[language=Python, basicstyle=\footnotesize\ttfamily]
class Diarisor:
    def diarise(self, input_file, output_file):
        print("Diarise method not implemented!")
        return PyannoteReturnCodes.NOT_IMPLEMENTED

    def embed(self, input_file, from_time, to_time):
        print("Embed method not implemented!")
        return PyannoteReturnCodes.NOT_IMPLEMENTED

class pyannote(Diarisor):
    def diarise(self, input_file, output_file):
        # Model specific diarise implementation

    def embed(self, input_file, from_time, to_time):
        # Model specific embed implementation
\end{lstlisting}
\end{frame}

\begin{frame}{Agenda Point Segmentation}
    \begin{itemize}
        \item Research Question 3
        \begin{itemize}
            \item Accurate segmentation of video into topics
        \end{itemize}
        \item LLM-Based segmentation
        \begin{itemize}
            \item Label important topics
            \item Identify start times using sliding window
        \end{itemize}
    \end{itemize}
\end{frame}

\begin{frame}{Information Retrieval}
    \begin{itemize}
        \item Research Question 4
        \begin{itemize}
            \item State-of-the-art techniques for information retrieval
        \end{itemize}
        \item Weaviate vector database client
        \begin{itemize}
            \item Vector search, BM25 search, Hybrid search
            \item Transcript collection
                \begin{itemize}
                    \item Speaking turn embedding
                    \item Voice profile embedding
                \end{itemize}
            \item Speaker collection
                \begin{itemize}
                    \item Voice profile embedding
                    \item Voice profile speaker name
                \end{itemize}
        \end{itemize}
    \end{itemize}
\end{frame}

\begin{frame}{Retrieval Augmented Generation}
    \begin{itemize}
        \item Research Question 5
        \begin{itemize}
            \item Large language model chat bot
        \end{itemize}
        \item Weaviate connection
        \begin{itemize}
            \item BM25 search
            \item Query + context > answer
        \end{itemize}
    \end{itemize}
\end{frame}

\begin{frame}\frametitle{RAG pipeline}
\begin{figure}
    \centering
    \includegraphics[width=0.9\textwidth]{images/rag.png}
    \caption{Retrieval Augmented Generation pipeline}
    \label{fig:rag}
\end{figure}
\end{frame}

\section{System architecture}
\begin{frame}\frametitle{Analysis pipeline}
\begin{figure}
    \centering
    \includegraphics[width=0.7\textwidth]{images/pipeline.png}
    % \caption{Video analysis pipeline}
    \label{fig:pipeline}
\end{figure}
\end{frame}

\section{Demo}
% \subsection{Analysis}
\begin{frame}\frametitle{Demo}
\end{frame}


\section{Conclusion}

\begin{frame}\frametitle{Conclusion}
\begin{itemize}
    \item \textit{How can AI be utilized in order to increase information retrievability of large video archives, in particular of democratically elected councils?}
    \begin{itemize}
        \item iBabs \& NotuBiz scrape tools
        \item Speaker diarisation \& automatic speech recognition
        \item Agenda topic segmentation
        \item Vector search, BM25 keyword search and hybrid search
        \item Chat bot
    \end{itemize}
\end{itemize}
\end{frame}

%-------------------------------------------------------------------------------
\end{document}
