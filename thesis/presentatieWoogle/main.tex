%-------------------------------------------------------------------------------
% LATEX TEMPLATE PRESENTATIE
%-------------------------------------------------------------------------------
% Dit template is voor gebruik door studenten van de de bacheloropleiding 
% Informatica van de Universiteit van Amsterdam.
% Voor informatie over presenteren, zie 
%                               https://practicumav.nl/presenteren/index.html
%
%-------------------------------------------------------------------------------
%	PACKAGES EN CONFIGURATIE
%-------------------------------------------------------------------------------
% Gebruik de optie "sidebar" voor toevoeging van een sidebar met inhoudsopgave
% en de optie "dyslexic" voor gebruik van het OpenDyslexic-lettertype
\documentclass[aspectratio=169,sidebar]{uva-inf-presentation}
\usepackage[english]{babel}
\usepackage{listings}

% Vul hieronder de gevraagde gegevens in, 
% meerdere auteurs en UvAnetID's gescheiden door een puntkomma 
\title{AI-Driven Retrieval and Analysis of Videotaped Council Meeting Archives using Automatic Speech Recognition, Speaker Identification, and Retrieval Augmented Generation} 
\authors{Pepijn van WIjk}
\uvanetids{13952072}
\course{Bachelor thesis}
\tutor{}
\docent{Dr. Maarten Marx} 
\group{}
\programme{Informatica}

\begin{document}
%-------------------------------------------------------------------------------
%	AUTOMATISCHE SLIDES
%-------------------------------------------------------------------------------
\begin{titelframe}
\titlepage
\end{titelframe}

\begin{titelframe}\frametitle{Contents}
\tableofcontents
\end{titelframe}

%-------------------------------------------------------------------------------
%	PRESENTATIE SLIDES
%------------------------------------------------------------------------------

\section{Problem}
\begin{frame}{Problem}
    \begin{itemize}
        \item Wet Open Overheid
        \item Tweede Kamer publishes reports
        \item Difficult for hundreds of local governments
        \item journalist writing about a local meeting decision
    \end{itemize}
\end{frame}

\section{Solution}
\begin{frame}{Solution}
    \begin{itemize}
        \item Automatic archiving tool
        \item Automatic analysis of meetings
        \begin{itemize}
            \item Transcription
            \item Speaker diarisation
        \end{itemize}
        \item Transcript search engine
        \item Conversation-style chat bot
        \item Easy navigable web application
    \end{itemize}
\end{frame}

\section{Methodology}
\begin{frame}\frametitle{Architecture}

\end{frame}

\begin{frame}\frametitle{Architecture - scraper}

\end{frame}

\begin{frame}\frametitle{Scraper}
\begin{itemize}
    \item iBabs \& NotuBiz
    \item Downloads video 
    \item Downloads agenda when available
    \item Organizes archive based on year and meeting type
\end{itemize}
\end{frame}

\begin{frame}\frametitle{Architecture}

\end{frame}

\begin{frame}\frametitle{Architecture - analysis}

\end{frame}

\begin{frame}\frametitle{Analysis pipeline}
\begin{itemize}
    \item
\end{itemize}
\end{frame}

\begin{frame}\frametitle{Architecture}

\end{frame}

\section{Demo}
\begin{frame}\frametitle{Demo}
\end{frame}

\section{Results}
\begin{frame}\frametitle{Results}

\end{frame}

\section{Conclusion}
\begin{frame}\frametitle{Conclusion}
\begin{itemize}
    \item Hard to navigate and find through multi-hour long municipal meetings
    \begin{itemize}
        \item Automatic iBabs \& NotuBiz scrape tools
        \item Speaker diarisation \& automatic speech recognition
        \item Semantic- and keyword search engine
        \item Chat bot
        \item Online search application
    \end{itemize}
\end{itemize}
\end{frame}

%-------------------------------------------------------------------------------
\end{document}
